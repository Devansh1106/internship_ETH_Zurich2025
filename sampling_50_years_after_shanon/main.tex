%% Credits of this ams template are with respective people. @Devansh1106 neither own this template nor the credits. 
\documentclass[reqno,9pt]{amsart}

\usepackage[numbers]{natbib}
\setlength{\bibsep}{8pt}
\usepackage{graphicx}
\usepackage{hyperref}
\usepackage{lineno}
\usepackage{amssymb}
\usepackage{amsmath}
\usepackage{amsthm}
\usepackage{mathtools}
\usepackage{mathrsfs}
\usepackage{esint}
\usepackage{fancyhdr}
\setlength{\parskip}{0.5\baselineskip}%
\setlength{\parindent}{20pt}
\theoremstyle{plain}

\pagestyle{fancy}
\fancyhf{} % Clear default headers/footers
\setlength{\headheight}{10.0pt}
% Even pages: Paper title
\fancyhead[LE]{\footnotesize\textit{Nonlocality and Nonlinearity Implies Universality}}

% Odd pages: Author name
\fancyhead[RO]{\footnotesize\textit{Devansh Tripathi}}

\newtheorem*{thm*}{Theorem}
%% this allows for theorems which are not automatically numbered

\renewcommand{\qedsymbol}{$\blacksquare$}
\newtheorem{thm}{Theorem}
\newtheorem{lem}{Lemma}
\newtheorem{prop}{Proposition}
\theoremstyle{definition}
\newtheorem{defn}{Definition}
\newtheorem{eg}{Example}
\newtheorem{rem}{Remark}
\newcommand{\bb}[1]{\mathbb{#1}}
\newcommand{\cal}[1]{\mathcal{#1}}

\hypersetup{
    colorlinks=false,
    linkcolor=blue,    % Internal links (sections, equations)
    citecolor=red,     % Citation links
    urlcolor=magenta   % URLs (DOIs, websites)
}
%% The above lines are for formatting.  In general, you will not want to change these.

\title{Report on the paper ``Sampling -- 50 Years After Shanon''}
\author{Devansh Tripathi$^1$ \\ ETH Z\lowercase{\"urich}}
\thanks{$^1$Seminar für Angewandte Mathematik, HG E 62.2, Rämistrasse 101, 8092 Zürich, Switzerland \\ \href{mailto:devansh.tripathi@sam.math.ethz.ch}{\texttt{devansh.tripathi@sam.math.ethz.ch}}}

\begin{document}
\numberwithin{equation}{section}

\begin{abstract}
    This document contains the important notes taken from the paper \cite{MU2000}. The emphasis is on the {\it regular} sampling, where the grid is uniform. This document introduce the reader to the modern, Hilbert-space formulation, we reinterpret Shanon's sampling procedure as an orthogonal projection onto the subspace of band-limited functions. Then the standard sampling paradigm is extended for a representation of functions in the more general class of ``shift-invariant'' function spaces, including splines and wavelets.  Practically, this allows for simpler and possibly more realistic—interpolation models, which can be used in conjunction with a much wider class of (anti-aliasing) prefilters that are not necessarily ideal low-pass. The report summarizes and discuss the results available for the determination of the approximation error and of the sampling rate when the input of the system is essentially arbitrary; e.g., nonbandlimited.

    \vspace{1em}
    \paragraph{\bf Keywords} Bandlimited, Hilbert spaces, Anti-aliasing.
\end{abstract}
\maketitle
\section{\bf \large Introduction}
    In 1949, Shannon published the paper “Communication in
    the Presence of Noise,” which set the foundation of information theory. n order to formulate his rate/distortion theory, Shannon needed a general mechanism for converting an analog signal into a sequence of numbers. This led him to state the classical sampling theorem at the very beginning of his paper in the following terms:
    \begin{thm}[Shannon]
        If a function $f(x)$ contains no frequencies higher than $\omega_{max}$(in radians per second), it is completely determined by giving its ordinates at a series of points spaced $T = \pi/\omega_{max}$ seconds apart.
    \end{thm}
    The reconstruction formula that complements the sampling theorem is 
    \begin{equation}\label{eqrf}
        f(x) = \sum_{k\in \bb Z} f(kT) sinc(x/T-k)
    \end{equation}
    in which the equidistant samples of $f(x)$ may be interpreted as coefficients of some basis functions obtained by appropriate shifting and rescaling of the sinc-function: $sinc(x) = \sin(\pi x)/(\pi x)$. Formula \ref{eqrf} is exact if $f(x)$ is bandlimited to $\omega_{max} \leq \pi/T$; this upper limit is the Nyquist frequency, a term coined by Shannon. In mathematical literature \ref{eqrf} is known as cardinal series expansion attributed to Whittaker \cite{Whittaker_1915,PLB1983}. Sampling theorem tells us how to convert an analog signal into a sequence of numbers, which can then be processed digitally or coded on a computer. 

    \paragraph{\bf Issues associated with Shanon's result}
    \begin{itemize}
        \item It is an idealization; real world signals or images are never exactly bandlimited.
        \item There is no such device as an ideal (anti-aliasing or reconstruction) low-pass filter.
        \item Shanon's reconstruction formula is rarely used in practice due to the slow decay of the sinc function. Instead, much simpler techniques such as linear interpolation are used.
    \end{itemize}
    \paragraph{\bf What is aliasing and anti-aliasing?} {\bf Aliasing:} When a signal is sampled, frequencies above the Nyguist frequency ``fold back'' into the sampled signal as lower frequencies, corrupting the data. A low-pass filter (LPF) with a cutoff is applied to eliminate these problematic frequencies. It removes the high signals above a cutoff and allows the signals lower than that cutoff to pass.

    \paragraph{\bf Anti-Aliasing:} Anti-aliasing is the process of removing or attenuating high-frequency components from a signal before sampling to prevent aliasing artifacts.

    
\bibliographystyle{plainnat}
\bibliography{ref}
\end{document}